%bra paket
\documentclass[twocolumn]{article}
\usepackage[utf8]{inputenc}
%\usepackage[swedish]{babel}
\usepackage{fancyhdr}
%\usepackage{times}
%\usepackage{alltt} %verbatim text med möjlighet till andra latexkommandon i.
%\usepackage[usenames,dvipsnames]{color} %fler färger att välja på
%\usepackage{wrapfig} %figurer som ligger sida vid sida med texten
%\usepackage[table]{xcolor} %bakgrundsfärg i tabeller
%\usepackage[small,compact]{titlesec} %Spara plats!!
\usepackage{amsmath}
\usepackage{multicol}
\usepackage{graphicx}
\usepackage{float} %gör så att man kan placera bilder exakt mha [H]
\usepackage[table]{xcolor} %bakgrundsfärg i tabeller

%\usepackage[ddmmyyyy]{datetime}

\usepackage{setspace}
%\usepackage[usenames,dvipsnames]{color} %fler färger att välja på
%\usepackage{pdfpages} %för att kunna använda includepdf i appendix
\usepackage{pgf}
\usepackage{tikz}
\usetikzlibrary{arrows,automata}
\usetikzlibrary{positioning}
\tikzset{
    state/.style={
           rectangle,
           draw=black, thick,
           minimum height=3em,
           inner sep=5pt,
           text centered,
           },
}

% Different font in captions
\newcommand{\captionfonts}{\em}
\makeatletter  % Allow the use of @ in command names
\long\def\@makecaption#1#2{%
  \vskip\abovecaptionskip
  \sbox\@tempboxa{{\captionfonts #1: #2}}%
  \ifdim \wd\@tempboxa >\hsize
    {\captionfonts #1: #2\par}
  \else
    \hbox to\hsize{\hfil\box\@tempboxa\hfil}%
  \fi
  \vskip\belowcaptionskip}
\makeatother   % Cancel the effect of \makeatletter


%marginaler
\setlength\topmargin{0in}
\setlength\headheight{11pt}
\setlength\textheight{8.1in}
\setlength\textwidth{6.5in}
\setlength\oddsidemargin{0in}
\setlength\evensidemargin{0in}
\setlength\parindent{0in}
\setlength\parskip{0in}
\frenchspacing %Oui!

%För att kunna typsätta delar för sig!
\newcommand{\master}{}

%då kör vi


\begin{document}
%%%%%%%%%%%%%%%%%%% Försättsblad %%%%%%%%%%%%%%%%%%%%%%%%
\begin{titlepage}
\title{\textbf{Electric lab-assistant} \\
\Large{Engineering Applications using Matlab}\\
\large{TNG016}}
\author{
\vspace{30pt}\\
\large
ED3:\bigskip \\
\begin{tabular}{l l}
	Dan	Helgesson & danhe046 \\
	Albert Skog	& albsk635 \\
	Karl Westerberg	& karwe772 \\
\end{tabular}\vspace{40pt}\\
Examiner: Qingxiang Zhao 
}
\date{Submitted: \today}
\maketitle
\thispagestyle{empty}
\begin{center}


\begin{figure}[b]
	\begin{center}
		\includegraphics[scale=0.6]{Figure/LIU-logo.jpg}
	\end{center}
\end{figure}

\end{center}

\end{titlepage}

%%%%%%%%%%%%%%%%%%% Header %%%%%%%%%%%%%%%%%%%%%%%
\pagestyle{fancy}
\fancyhead[l]{Engineering Applications using Matlab\\Electric Lab-Assistant}
\fancyhead[r]{Dan Helgesson, Albert Skog\\ \& Karl Westerberg}
\fancyfoot[c]{}
%%%%%%%%%%%%%%%%%%% Contents %%%%%%%%%%%%%%%%%%%%%
\onecolumn
\tableofcontents
\clearpage
\setcounter{page}{1}
\fancyfoot[c]{\thepage}
\twocolumn


%%%%%%%%%%%%%%%%%% Rapporten %%%%%%%%%%%%%%%%%%%%%
\section{Introduction}
%Om filen typsätts som del av hela rapporten så finns \master definierat i början och ingen \begin{document} och \end{document} får finnas, men för att kunna typsätta filen för sig är dem ett måste! \newcommand{\master}{} krävs i början på huvudrapporten!
\ifdefined\master
\else
	\documentclass[twocolumn]{article}
	%\input{../preamble}
	\begin{document}
\fi
The personal computer is at the center of attention for a large portion of our daily lives. In electronics education and research it is used to write code, design and simulate circuits, take measurments and to write about the the results. This project focuses on the borderlands between the latter two; connecting lab equipment and results presentation.

\subsection{Background}
Doing lab work often includes tedious, repetitive and time-consuming tasks. Some times good solutions are lacking, for example when a large series of values need to be measured at different signal frequencies, or when students are encouraged to take photographs of the oscilloscope screen for presenting in lab reports.

\subsection{Purpose}
The goal of this project was to investigate weather Matlab could be used to create a common interface to speed up and streamline electronic lab tasks and documentation.

\subsection{Method}
Matlabs tools for graphical user interfaces were used in conjuction with the GPIB-bus to create software for controling both input and output of an experiment, aswell as  






\ifdefined\master
\else
	\end{document}
\fi

\section{Lab-Assistant}

The user interface of Lab-Assistant is designed to be intuitive and user-friendly. When launched the user is met by three empty columns, each with a popup-menu at the top. The columns are called \emph{input}, \emph{output} and \emph{export}, each with a set of 3-4 different options. Using different combinations of these options, a variety of tasks can be performed.
%Om filen typsätts som del av hela rapporten så finns \master definierat i början och ingen \begin{document} och \end{document} får finnas, men för att kunna typsätta filen för sig är dem ett måste! \newcommand{\master}{} krävs i början på huvudrapporten!
\ifdefined\master
\else
	\documentclass[twocolumn]{article}
	\usepackage{graphicx}
	\usepackage{float}
	%\input{../preamble}
	\begin{document}
\fi

\subsection{Inputs}
Lab Assistant supports two types of devices for input; function generators and voltage generators. The function generator can also be used to generate a frequency sweep.

\subsubsection*{Function Generator}
Uses function generator to output a signal. The \emph{waveform} popup-menu sets the signal type to sine, square, triangle or sawtooth. If no waveform is selected the current waveform of the function generator is used. \emph{Frequency}, \emph{amplitude} and \emph{offset} controlls the parameters of the output. In order to establish a connection to the function generator its GPIB-address must be put into the \emph{GPIB-address} field. All options are shown in Figure \ref{fig:funcgen}.

\begin{figure}[H]
\centering
\fbox{\includegraphics[width=4cm]{Figure/function_generator.png}}
\caption{Function Generator settings.}
\label{fig:funcgen}
\end{figure}

\subsubsection*{Frequency Sweep}
The function generator can also be used to generate a Bode plot of a connected system. When \emph{frequency sweep} is selected as input, output is automatically set to \emph{bode graph} and no other outputs can be chosen. Available settings for the sweep are \emph{start frequency}, \emph{end frequency}, \emph{step length}, \emph{amplitude} and are shown in Figure \ref{fig:freqsw}. GPIB-address must allso be filled in as described above.

\begin{figure}[H]
\centering
\fbox{\includegraphics[width=4cm]{Figure/frequency_sweep.png}}
\caption{Frequency sweep settings.}
\label{fig:freqsw}
\end{figure}

\subsubsection*{Voltage Generator}
The other device supported is the voltage generator. Apart from \emph{GPIB-Address} of the voltage generator, the available settings are \emph{voltage} and \emph{current limit}. Output can be toggled with the \emph{output on/off} pushbutton. These options are shown in Figure \ref{fig:voltgen}.

\begin{figure}[H]
\centering
\fbox{\includegraphics[width=4cm]{Figure/voltage_generator.png}}
\caption{Voltage Generator settings.}
\label{fig:voltgen}
\end{figure}

\ifdefined\master
\else
	\end{document}
\fi
%Om filen typsätts som del av hela rapporten så finns \master definierat i början och ingen \begin{document} och \end{document} får finnas, men för att kunna typsätta filen för sig är dem ett måste! \newcommand{\master}{} krävs i början på huvudrapporten!
\ifdefined\master
\else
	\documentclass[twocolumn]{article}
	%\input{../preamble}
	\begin{document}
\fi

\subsection{Outputs}
Two methods of gathering output data are available

\subsubsection*{Oscilloscope}
The oscilloscope has two functions, picture and measurement. The picture option practicaly takes a copy of the oscilloscope screen and shows it in the panel. The measurement option is used for getting the graph and plot it without all extra information the oscilloscope shows. The plot is then displayed in the panel. After entering the \emph{GPIB address} for the oscilloscope, the \emph{Start} button will execute the option chosen.

\subsubsection*{Multimeter}
The multimeter function is just a multimeter, that displays the value in the panel. It's not possible to store the value as it is with a picture.

\subsubsection*{Bode Graph}
When the frequency sweep is chosen in the input panel it automaticaly chose the bode graph in the output panel. The Bode Graph only works with the frequency sweep as the start button is placed in the input panel. The plot is shown in the output panel.



\ifdefined\master
\else
	\end{document}
\fi
%Om filen typsätts som del av hela rapporten så finns \master definierat i början och ingen \begin{document} och \end{document} får finnas, men för att kunna typsätta filen för sig är dem ett måste! \newcommand{\master}{} krävs i början på huvudrapporten!
\ifdefined\master
\else
	\documentclass[twocolumn]{article}
	\usepackage{graphicx}
	\usepackage{float}
	%\input{preamble}
	\begin{document}
\fi
\subsection{Exports}
The program has four different choices for exporting the obtained data; copy to clipboard, save as image, create LaTeX report and create Microsoft Word report.

\subsubsection*{Clipboard}
Copies the figure to clipboard. Textboxes enable the user to modify title and axis labels before copying. (Figure \ref{fig:funcgen})

\begin{figure}[H]
\centering
\fbox{\includegraphics[width=4cm]{Figure/clipboard.png}}
\caption{Copy to clipboard export settings.}
\label{fig:clip}
\end{figure}

\subsubsection*{Image}
Saves the image to disk. Textboxes enable the user to modify title and axis labels before copying. (Figure \ref{fig:img})

\begin{figure}[H]
\centering
\fbox{\includegraphics[width=4cm]{Figure/image.png}}
\caption{Image export settings.}
\label{fig:img}
\end{figure}

\subsubsection*{LaTeX}
Generates LaTeX report (Appendix \ref{output}) containing the figure. \emph{Image name} textbox enables user to choose file name and format for the figure, which will be saved at the same location as the tex-file. \emph{Caption}, \emph{label} and \emph{width} properties are transferred to corresponding LaTeX commands, while tite and labels are applied directly to the figure before exporting. (Figure \ref{fig:funcgen})

\begin{figure}[H]
\centering
\fbox{\includegraphics[width=4cm]{Figure/latex.png}}
\caption{LaTeX export settings.}
\label{fig:tex}
\end{figure}

\subsubsection*{Word}
Generates Microsoft Word report containing the figure. \emph{Title}, \emph{x-label} and \emph{y-label} overrides the settings of the figure and the caption is inserted on a centred line below. (Figure \ref{fig:word})

\begin{figure}[H]
\centering
\fbox{\includegraphics[width=4cm]{Figure/word.png}}
\caption{Word export settings.}
\label{fig:word}
\end{figure}

\subsubsection*{Workspace}
The measurment data can be saved to a .mat-file. This export option has no settings other than save-location.

\ifdefined\master
\else
	\end{document}
\fi

\section{Examples}

\section{Conclusion}
%Om filen typsätts som del av hela rapporten så finns \master definierat i början och ingen \begin{document} och \end{document} får finnas, men för att kunna typsätta filen för sig är dem ett måste! \newcommand{\master}{} krävs i början på huvudrapporten!
\ifdefined\master
\else
	\documentclass[twocolumn]{article}
	\input{../preamble}
	\begin{document}
\fi
%text goes here!









\ifdefined\master
\else
	\end{document}
\fi

\end{document}